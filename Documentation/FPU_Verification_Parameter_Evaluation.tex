% -*- mode: latex; eval: (flyspell-mode 1) -*-
\documentclass[11pt,a4paper,twoside]{scrartcl}


%\usepackage{minted}
%\usepackage{euler}
%\usepackage{concrete}
\usepackage{hyperref}
\usepackage{graphicx}
\usepackage{xcolor}
\usepackage{framed}
\usepackage{gensymb}
\usepackage{cmbright}
\usepackage{lmodern}
\usepackage[T1]{fontenc}
\usepackage{makeidx}
\usepackage{amsmath}
\DeclareMathOperator{\round}{round}
\DeclareMathOperator{\std}{std}
\DeclareMathOperator{\RSS}{RSS}
\makeindex


\pagestyle{headings}
\begin{document}

\title{Specification of Parameter evaluations in the automated FPU verification software}

\author{Johannes Nix}

\maketitle
\tableofcontents

\cleardoublepage
\section{Purpose of document}

The specification of the verification system contains some notes on
desired parameter evaluations. These are in non-formal language and
leave room to interpretation and even misunderstandings.  Correcting
the evaluation can be potentially extremely time-consuming, if
required input data needs to be captured anew.

This document has the purpose to define the parameter evaluations in a
formal description so that time-consuming re-runs of FPU measurements
can be avoided.

\section{Used symbols}
\begin{description}
\item[$ \left< \{ x_1, x_2, x_3, \ldots \} \right> $] Arithmetic
  mean of values in the set of values $\{ x_n \} $ (where
  $x_n$ can be a scalar or a vector).
\item[$\max(\{x_1, x_2, x_3, \ldots \})$] Maximum element of
  the set of scalar values $\{ x_n \}$.
\item[$\std(\{x_1, x_2, x_3, \ldots \})$] Standard deviation of the
  elements in the the set $\{ x_n \}$.
\item[$\RSS(\{x_1, x_2, x_3, \ldots \})$] "Root Sum Square" value of
  elements in the the set $\{ x_n \}$, defined as
  \[
  \RSS(\{x_1, x_2, x_3, \ldots\}) = \sqrt{x_1^2 + x_2^2 + x_3^2 + \ldots}
  \]

\item[$I_{\alpha,\beta}$] Image of FPU captured at arm coordinates
  $\alpha$ and $\beta$.
\item[$\mathcal{L(I)}$] Function which returns vector coordinates of
  the large metrology target of an FPU.
\item[$\mathcal{S(I)}$] Function which returns vector coordinates of
  the small metrology target of an FPU.
\item[$\mathcal{F(I)}$] Function which returns vector coordinates of
  the illuminated fibre of an FPU.
\item[$\mathcal{P(I)}$] Function which returns vector coordinates of
  blob center in a pupil alignment image.
  \item[$||\vec{a}||$] Magnitude or length of vector $\vec{a}$
\end{description}

All distances are entered and displayed in millimeters, and all angles
are displayed in degrees.

\section{Metrology target calibration}

This is specified in the specification document in section 5.1.
\subsection{Symbols}

\begin{description}
\item[$d_l$] distance from fibre to large metrology target
\item[$d_s$] distance from fibre to small metrology target
\item[$\angle_{lfs}$] angle from large tagret, to fibre, to small target
\item[$I_M$] is the corresponding metrology target image
\item[$I_F$] is the corresponding image of the fibre
\end{description}

\subsection{Definitions}

\begin{eqnarray}
  \vec{a} & =  & \mathcal{L(I_M)} - \mathcal{F(I_F)} \\
  d_{lf} & = &  ||\vec{a}||
\end{eqnarray}

\begin{eqnarray}
  \vec{b} & = & \mathcal{S(I_M)} - \mathcal{F(I_F)} \\
  d_{sf} & =  & ||\vec{b}||
\end{eqnarray}

\begin{eqnarray}
  \angle_{lfs} & = &  \sin^{-1}( \frac{\vec{a} \times \vec{b}}{||\vec{a}|| \cdot ||\vec{b}||} )
\end{eqnarray}


\section{Pupil alignment}
\subsection{Symbols}
\begin{itemize}
  \item[$I_{p,\alpha,\beta}$] pupil alignment image for arm angles $(\alpha,\beta)$
\item[$\mathcal{P}(I_{p,\alpha,\beta})$] function which returns blob center of pupil alignment image
\item[$\vec{c} = (x_c, y_c)$] calibrated central coordinate (set as configuration parameter)
\item[$\vec{s_{\alpha,\beta}}$]  coordinates of spot for angle coordinates $(\alpha, \beta)$
\item[$\{ \alpha_1, \alpha_2, \alpha_3, \alpha_4\}$] measured alpha coordinates with values -170\degree, -80\degree, 10\degree, 100\degree degrees
\item[$\{ \beta_1, \beta_2, \beta_3, \beta_4\}$] measured beta coordinates  with values -170\degree, -80\degree, 10\degree, 100\degree degrees
\item[$\epsilon_\alpha$] mean error for alpha variation
\item[$\epsilon_\beta$] mean error for beta variation
\item[$L_f$] instrument focal length or radius
\item[$\epsilon_\mathrm{chassis}$] chassis error
\item[$\epsilon_\mathrm{tot}$] total error

\end{itemize}

\subsection{Definitions}
We define two level of center points. The first is when only $\beta$ is varied,
and the beta arm is moved around one position. This is used to derive
the beta error value:

\begin{eqnarray}
  \vec{s}_{\alpha_j,\beta_k} & =  & \mathcal{P}(I_{p,\alpha_j,\beta_k}) \\
  \vec{u}_{\mathrm{avg}, \alpha_j} & = & \frac{1}{4} \sum_{k=1}^{4} \vec{s}_{\alpha_j,\beta_k} = \left<\vec{s}_{\alpha_j,\beta_k}, k = 1 \ldots 4 \right>\\
  \epsilon_\beta & = & \left< ||\vec{s}_{\alpha_j,\beta_k} - \vec{u}_{\mathrm{avg}, \alpha_j} || : j = 1 \ldots 4, k = 1 \ldots 4 \right>
\end{eqnarray}

The second level of center points averages over the beta center points
to get an alpha center point. This is used to derive the alpha error
value.
\begin{eqnarray}
  \vec{v}_\mathrm{avg} & = & \left< \vec{u}_{\mathrm{avg}, \alpha_j}, j = 1 \ldots 4 \right> \\
  \epsilon_\alpha & = & \left< ||\vec{s}_{\mathrm{avg}, \alpha_j} - \vec{v}_\mathrm{avg} || : j = 1 \ldots 4 \right>
\end{eqnarray}

The average of the error over all points is converted to an angle,
which is the chassis error:
\begin{eqnarray}
  \vec{x_c} & = &  \left< \vec{s}_{\alpha_j,\beta_k} - \vec{c} : j = 1 \ldots 4, k = 1 \ldots 4 \right> \\
  \epsilon_\mathrm{chassis} & = & \tan^{-1}\left(\frac{||x_c||}{L_f}\right)
\end{eqnarray}



\section{Datum repeatability}
\subsection{Symbols}
\begin{itemize}
\item[$I_{\mathrm{only},j}$] Image of FPU captured after datum only
\item[$I_{\mathrm{moved},k}$] Image of FPU captured after move, then datum
\item[$\mathcal{L(I)}$] Function which returns vector coordinates of
  the large metrology target of an FPU.
\item[$\mathcal{S(I)}$] Function which returns vector coordinates of
  the small metrology target of an FPU.

\item[$\epsilon_\mathrm{max,only}$] maximum error for datum-only operation
\item[$\sigma_\mathrm{max,only}$] standard deviation for datum-only operation
\item[$\epsilon_\mathrm{max,moved}$] maximum error for move-then-datum operation
\item[$\sigma_\mathrm{max,moved}$] standard deviation for move-then-datum operation


\end{itemize}
\subsection{Definitions}

First, two baseline points for all measurements are defined:
\begin{eqnarray}
  \vec{l_c} & =  & \left< \mathcal{L(I)}, \mathrm{ all images} \right> \\
  \vec{s_c} & =  & \left< \mathcal{S(I)}, \mathrm{ all images} \right> \\
\end{eqnarray}

These are used to derive the error distances:
\begin{eqnarray}
  d_{\mathrm{large}, \mathrm{only}, j} & = & || \mathcal{L}(I_{\mathrm{only},j}) - \vec{l_c} ||\\
  d_{\mathrm{small}, \mathrm{only}, k} & = & || \mathcal{S}(I_{\mathrm{only},j}) - \vec{s_c} ||\\
  d_{\mathrm{large}, \mathrm{moved}, j} & = & ||  \mathcal{L}(I_{\mathrm{moved},j}) - \vec{l_c}|| \\
  d_{\mathrm{small}, \mathrm{moved}, k} & = & || \mathcal{S}(I_{\mathrm{moved},j}) - \vec{s_c} ||
\end{eqnarray}

We define two sets which contain the error distances for both the
large and the small metrology target:
\begin{eqnarray}
  D & = & \{d_{\mathrm{large}, \mathrm{only}, j} | k = 1 \ldots, d_{\mathrm{small}, \mathrm{only}, j} | j = 1 \ldots \} \\
  M & = & \{d_{\mathrm{large}, \mathrm{moved}, j | k = 1 \ldots}, d_{\mathrm{small}, \mathrm{moved}, j}    | j = 1 \ldots \}
\end{eqnarray}

With these sets, these distances are used to define maximum errors and standard deviations for both
the datumed-only, and the datumed-then-moved case:
\begin{eqnarray}
  \epsilon_\mathrm{max,only} & = & \max(\{d | d \in D\}) \\
  \epsilon_\mathrm{max,moved} & = & \max(\{m | m \in M\}) \\
  \sigma_\mathrm{only} & = & \std(\{d | d \in D\}) \\
  \sigma_\mathrm{moved} & = & \std(\{m | m \in M\})
\end{eqnarray}

The estimate of the standard deviation should use Bessel's correction,
as the mean which is used internally, is a statistical estimate itself.

Estimating percentile values might be of interest for drawing
conclusions about reliability.

\section{Positional repeatability}
\subsection{Symbols}
\begin{itemize}
  \item[]
\item[$I_{\alpha_i,\beta_i,i}$] Image of FPU captured after move to position $(\alpha_i,\beta_i)$
\item[$\mathcal{L(I)}$] Function which returns vector coordinates of
  the large metrology target of an FPU.
\item[$\mathcal{S(I)}$] Function which returns vector coordinates of
  the small metrology target of an FPU.
\item[$A = \{ \alpha_j\}$] set of all measured alpha angles
\item[$B = \{ \beta_k\}$] set of all measured beta angles
  \item[$\epsilon^{a}_\alpha$] maximum positional error (distance) for images with a common
    alpha position of $\alpha$.
  \item[$\epsilon^{b}_\beta$] maximum positional error (distance) for images with a common
    beta position of $\beta$.
    \item[$\epsilon_\mathrm{all}$] largest positional error distance,
      over all images taken
\end{itemize}

\subsection{Definitions}
For each value $\alpha \in A$ and $\beta \in B$, there is a set of all
pictures which were taken at that angle:
\begin{eqnarray}
  S^a_\alpha & = &  \{ I_{\alpha',\beta,i} | \alpha' = \alpha \} \\
  S^b_\beta & = &  \{ I_{\alpha,\beta',i} | \beta' = \beta \}
\end{eqnarray}

Note that the pictures taken have the property that between two
pictures, at most only one of $\alpha$ or $\beta$ can be different;
this property is not used here.

Then, we define for each $\alpha \in A$ and for each $\beta \in B$
center coordinates, for both the small and the large metrology target
\[
\vec{p}_c^\mathrm{\,small}
\] and
\[
\vec{p}_c^\mathrm{\,large}
\]
of focal plane positions, by averaging over all images from a single
set (and which have consequently one coordinate with the same value, for
either $\alpha$ or $\beta$):
\begin{eqnarray}
  \vec{p}_{c,\alpha}^\mathrm{\,small,a} & = &  \left< \{ \mathcal{S}(I) | I \in S^a_\alpha  \} \right> \\
  \vec{p}_{c,\beta}^\mathrm{\,small,b}  & = &  \left< \{ \mathcal{S}(I) | I \in S^b_\beta \} \right> \\
  \vec{p}_{c,\alpha}^\mathrm{\,large,a} & = &  \left< \{ \mathcal{L}(I) | I \in S^a_\alpha \} \right> \\
  \vec{p}_{c,\beta}^\mathrm{\,large,b}  & = &  \left< \{ \mathcal{L}(I) | I \in S^b_\beta \} \right>
\end{eqnarray}

Note that these center points do \emph{not} correspond only to images
where the FPU was driven to the same $(\alpha, \beta)$ position;
rather, it is an average over all measurements where the FPU was at
one alpha value.

Using these center coordinates, we can compute error distances for
images from each set, and each associated center point:
\begin{eqnarray}
  d^\mathrm{\,small,a}_{\alpha,\beta,i} & = & ||\mathcal{S}(I_{\alpha,\beta,i}) - \vec{p}_{c,\alpha}^\mathrm{\,small,a} ||,  I_{\alpha,\beta,i} \in S^a_\alpha \\
  d^\mathrm{\,large,a}_{\alpha,\beta,i} & = & ||\mathcal{L}(I_{\alpha,\beta,i}) - \vec{p}_{c,\alpha}^\mathrm{\,large,a} ||,  I_{\alpha,\beta,i} \in S^b_\beta  \\
  d^\mathrm{\,small,b}_{\alpha,\beta,i} & = & ||\mathcal{S}(I_{\alpha,\beta,i}) - \vec{p}_{c,\beta}^\mathrm{\,small,b} ||,   I_{\alpha,\beta,i} \in S^a_\alpha \\
  d^\mathrm{\,large,b}_{\alpha,\beta,i} & = & ||\mathcal{L}(I_{\alpha,\beta,i}) - \vec{p}_{c,\beta}^\mathrm{\,large,b} ||,   I_{\alpha,\beta,i} \in S^b_\beta
\end{eqnarray}

and find the maximum distance within each set of values with a common element from $A$ or $B$, respectively:

\begin{eqnarray}
\epsilon^{a}_\alpha & = & \max(\{ d^\mathrm{\,small,a}_{\alpha',\beta,i}, d^\mathrm{\,large,a}_{\alpha',\beta,i} | \alpha' = \alpha \in A \}) \\
\epsilon^{b}_\beta & = & \max(\{ d^\mathrm{\,small,b}_{\alpha,\beta',i}, d^\mathrm{\,large,b}_{\alpha,\beta',i} | \beta' = \beta \in B \})
\end{eqnarray}

The specification defines
\begin{eqnarray}
  \mathrm{POSREP\_ALPHA\_MAX} & = &  \max(\{ \epsilon^{a}_\alpha | \alpha \in A \}) \\
  \mathrm{POSREP\_BETA\_MAX} & = & \max(\{ \epsilon^{b}_\beta | \beta \in B \})
\end{eqnarray}

For these $ S^a_\alpha $ and $S^b_\beta$ which only have one element,
the associated error distance is zero, which however does not affect
the parameters descfribed next.


The maximum overall error is then
\begin{equation}
  \epsilon_\mathrm{all} = \max(\{\epsilon^{a}_\alpha | \alpha \in A, \epsilon^{b}_\beta | \beta \in B \})
\end{equation}.

%% Note that
%% \begin{equation}
%%   \label{eq:maxmax}
%% \max(\{\epsilon^{a}_\alpha | \alpha \in A\}) = \max(\{\epsilon^{b}_\beta | \beta \in B\})
%% \end{equation}
%% because every element of an $S^a_\alpha$ is also an element of
%% $S^b_\beta$, and therefore the maximum of $\{ \epsilon^{a}_\alpha |
%% \alpha \in A \}$ is also the maximum of $\{ \epsilon^{b}_\beta | \beta
%% \in B \}$.
%%

For the overall error, the specification uses the RSS or root sum
square value.

The specification defines the value $\mathrm{POSREP\_RSS}$ as
\begin{eqnarray}
  \mathrm{POSREP\_RSS} & = &  \RSS( \max(\{\epsilon^{a}_\alpha| \alpha \in A\}), \max(\{\epsilon^{b}_\beta| \beta \in B\}) ) \\
  & = & \sqrt{\max(\{\epsilon^{a}_\alpha| \alpha \in A\})^2 + \max(\{\epsilon^{b}_\beta| \beta \in B\})^2 }
%%   & = & \sqrt{2 \cdot (\epsilon_\mathrm{all})^2} \\
%%   & = & \sqrt{2} \cdot \epsilon_\mathrm{all}
\end{eqnarray}
The only difference to the vector magnitude is that the input
arguments are distances, not coordinates from a coordinate system.


\section{Positional verification}
\subsection{Symbols}
\begin{itemize}
  \item[]
\item[$I_{\alpha_i,\beta_i}$] Image of FPU captured after move to position $(\alpha_i,\beta_i)$
\item[$\mathcal{L(I)}$] Function which returns vector coordinates of
  the large metrology target of an FPU.
\item[$\mathcal{S(I)}$] Function which returns vector coordinates of
  the small metrology target of an FPU.

\end{itemize}

\subsection{Definitions}

The positional verification computation is not yet defined.
The measurement moves the FPU to $N$ different positions,
and takes images

\[
I_{\alpha_i,\beta_i}, i = 1 \ldots N
\]
From these images, the coordinates in the focal surface are computed.
Possibly, these coordinates are projected on a plane surface,
because systematic aberrations matter here, in difference
to other verification steps.

From these images, the observed position
\[
\vec{x_i}
\]
is computed, and compared to a nominal position
\[
\tilde{\vec{x_i}} = f(\alpha_i,\beta_i)
\]
with some undefined function $f$. The nominal position is a function
of the angles, the motor axis geometry and some estimate of the alpha
and beta arm lengths, which could be derived by fitting or using the
metrology calibration measurement.

The length of the difference vector
\[
d_i = || \vec{x_i} - \tilde{\vec{x_i}}||
\]
is the positional error for this measurement.  The suggested extracted
measures are maximum and standard deviation of this error. It is also
possible to compute statistical deciles, which give information about
the distribution of errors. Within the verification system, it is only
planned to measure about 10 to 20 positions for the positional verification.

\begin{eqnarray}
\end{eqnarray}

\end{document}
